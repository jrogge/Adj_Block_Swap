%%%%%%%%%%%%%%%%%%%%%%%%%%%%%%%%%%%%%%%%%%%%%%%%%%%%%%%%%%%%%%%%%%%%%%%%%%%%%%%%
\documentclass[12pt]{article}
%\usepackage[a4paper, total={6.2in, 8.7in}]{geometry}
\usepackage{amsmath,amsthm,amssymb,amsfonts, enumitem, fancyhdr, color, comment,
graphicx, environ, parskip}
%\pagestyle{fancy}
%\setlength{\headheight}{15pt}
%%%%%%%%%%%%%%%%%%%%%%%%%%%%%%%%%%%%%%%%%%%%%%%%%%%%%%%%%%%%%%%%%%%%%%%%%%%%%%%%
\newcommand{\Z}{\mathbb{Z}}
\newcommand{\Q}{\mathbb{Q}}
\newcommand{\R}{\mathbb{R}}
\newcommand{\N}{\mathbb{N}}
\newcommand{\C}{\mathbb{C}}
\newcommand{\V}{\mathbb{V}}
\newcommand{\A}{\mathbb{A}}

\newtheoremstyle{pf}% name of the style to be used
{3pt}% measure of space to leave above the theorem. E.g.: 3pt
{3pt}% measure of space to leave below the theorem. E.g.: 3pt
{}% name of font to use in the body of the theorem
{}% measure of space to indent
{\normalfont\bfseries}% name of head font
{:}% punctuation between head and body
{0.5em}% space after theorem head; " " = normal interword space
{}% Manually specify head
\theoremstyle{pf}
\newtheorem*{mot}{Motivation}
\newtheorem*{op}{Operation}
\newtheorem*{prob}{Problem}
\newtheorem*{q}{Question}
\newtheorem*{pf}{Proof}

\title{Adjacent Block Interchange Problem}

\begin{document}

\maketitle

\section{Intro}

\begin{mot}
    You have a PDF but it's reversed! You can only move the pages by
    designating a range of pages and moving them to a specified location.
\end{mot}

\begin{op}
    Given a list, you may designate a range and a point in that range (between
    values) and the blocks on either side are (rigidly) swapped.
    Notation:
    \begin{align*}
        \mid_0 1 \mid_1 2 \mid_2 3 \mid_3 \ldots \mid_{n-1} n \mid_n
    \end{align*}
    Denote a swap of the range $a,c$ about point $b$ (s.t. $a < b < c$) by
    $[a,b,c]$.
\end{op}

\begin{prob}
    If the list is reversed, what is the optimal sequence of swaps to sort it?
\end{prob}
A clear upper bound is $n-1$: the sequence of swaps $s_i = [1, i-1, i]$ (for
$1 \le i < n$) reverses the list. Is this the best we can do?

\begin{q}
    If we permute the swaps above by some permutation $\sigma \in S_{n-1}$
    and associate to it the permutation defined on the reversal (so that
    $id \in S_{n-1}$ is sent to $id \in S_n$), what can we say about this map?
    It is certainly a map of sets $S_{n-1} \to S_n$ preserving the identity,
    but is it a group homomorphism? If so, what is the image?
\end{q}

\begin{q}
    In general, what is the best sorting algorithm using this operation?
\end{q}

\subsection{Duality}
Observe that if $\sigma$ is the reversing permutation $(n \ 1) (n-1 \ 2) \ldots$
then it is its own inverse. It is fixed under conjugation by itself and,
importantly, an operation of the type we are considering is sent to another
under conjugation by $\sigma$, so conjugation by $\sigma$ defines an involution
on the solution set. Call orbits of solutions under conjugation by $\sigma$
equivalence classes. Visually, this corresponds to flipping the entire
permutation. Algebraically an opertation $[a,b,c]$ on $n$ elements is sent
to $[n - c, n - b, n - a]$ (note the order).

\section{Optimal sequences}
\begin{center}
    \begin{tabular}{c|c|c}
        & \text{length} & \text{equivalence classes}\\
        \hline
        2 & 1 & 1\\
        \hline
        3 & 2 & 2\\
        \hline
        4 & 3 & 10\\
        \hline
        5 & 3 & 1\\
        \hline
        6 & 4 & 21\\
        \hline
        7 & 4 & 4\\
        \hline
        8 & 5 & 57\\
        \hline
        9 & 5 & 5\\
        \hline
        10 & 6 & 77\\
    \end{tabular}
\end{center}
%\begin{enumerate}
%    \setcounter{enumi}{1}
%    \item 1
%    \item 2
%    \item 3
%    \item 3
%    \item 4
%    \item 4
%    \item 5
%    \item 5
%    \item 6
%\end{enumerate}
%
%Number of equivalence classes of solutions
%\begin{enumerate}
%    \setcounter{enumi}{1}
%    \item 1
%    \item 2
%    \item 10
%    \item 1
%    \item 21
%    \item 4
%    \item 57
%    \item 5
%    \item ?? (code still running
%\end{enumerate}

(Lexicographically earliest)
\begin{enumerate}
    \setcounter{enumi}{1}
    \item $[[0,1,2]]$
    \item $[[0,1,3], [0,1,2]]$
    \item $[[0,1,4], [0,1,3], [0,1,2]]$
    \item $[[0,2,4], [1,3,5], [0,2,4]]$
    \item $[[0, 1, 2], [0, 3, 5], [1, 4, 6], [0, 2, 4]]$
    \item $[[0, 2, 5], [1, 4, 7], [0, 3, 5], [1, 4, 6]]$
    \item $[[0, 1, 2], [0, 3, 6], [1, 5, 8], [0, 3, 5], [1, 4, 6]]$
    \item $[[0, 2, 6], [1, 5, 9], [0, 4, 6], [1, 5, 7], [2, 6, 8]]$
    \item $[[0, 1, 2], [0, 3, 7], [1, 6, 10], [0, 4, 6], [1, 5, 7], [2, 6, 8]]$
\end{enumerate}

\subsection{Almost symmetric optimal sequences}
No self-dual solutions have been found yet. However,
for $n =6$, there are two nicely symmetric solutions.
%Neither are
%precisely self-dual but they are almost self-dual.
\begin{align*}
    [[0, 3, 6], [0, 2, 4], [1, 3, 5], [2, 4, 6]]\\
    [[0, 3, 6], [2, 4, 6], [1, 3, 5], [0, 2, 4]]\\
    \\
    [[1, 3, 5], [2, 4, 6], [1, 3, 5], [0, 1, 6]]\\
    [[1, 3, 5], [0, 2, 4], [1, 3, 5], [0, 5, 6]]
\end{align*}
\end{document}
